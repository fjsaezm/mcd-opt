\documentclass[a4paper]{article}

\addtolength{\hoffset}{-2.25cm}
\addtolength{\textwidth}{4.5cm}
\addtolength{\voffset}{-3.25cm}
\addtolength{\textheight}{5cm}
\setlength{\parskip}{0pt}
\setlength{\parindent}{0in}

%----------------------------------------------------------------------------------------
%	PACKAGES AND OTHER DOCUMENT CONFIGURATIONS
%----------------------------------------------------------------------------------------

\usepackage{blindtext} % Package to generate dummy text
\usepackage{charter} % Use the Charter font
\usepackage[utf8]{inputenc} % Use UTF-8 encoding
\usepackage{microtype} % Slightly tweak font spacing for aesthetics
\usepackage[english]{babel} % Language hyphenation and typographical rules
\usepackage{amsthm, amsmath, amssymb} % Mathematical typesetting
\usepackage{float} % Improved interface for floating objects
\usepackage[final, colorlinks = true,
            linkcolor = black,
            citecolor = black]{hyperref} % For hyperlinks in the PDF
\usepackage{graphicx, multicol} % Enhanced support for graphics
\usepackage{xcolor} % Driver-independent color extensions
\usepackage{marvosym, wasysym} % More symbols
\usepackage{rotating} % Rotation tools
\usepackage{censor} % Facilities for controlling restricted text
\usepackage{listings} % Environment for non-formatted code, !uses style file!
\usepackage{pseudocode} % Environment for specifying algorithms in a natural way
 % Environment for f-structures, !uses style file!
\usepackage{booktabs} % Enhances quality of tables
\usepackage{tikz-qtree} % Easy tree drawing tool
 % Configuration for b-trees and b+-trees, !uses style file!
\usepackage[backend=biber,style=numeric,
            sorting=nyt]{biblatex} % Complete reimplementation of bibliographic facilities
\addbibresource{ecl.bib}
\usepackage{csquotes} % Context sensitive quotation facilities
\usepackage[yyyymmdd]{datetime} % Uses YEAR-MONTH-DAY format for dates
\renewcommand{\dateseparator}{-} % Sets dateseparator to '-'
\usepackage{fancyhdr} % Headers and footers
\pagestyle{fancy} % All pages have headers and footers
\fancyhead{}\renewcommand{\headrulewidth}{0pt} % Blank out the default header
\fancyfoot[L]{} % Custom footer text
\fancyfoot[C]{} % Custom footer text
\fancyfoot[R]{\thepage} % Custom footer text
\newcommand{\note}[1]{\marginpar{\scriptsize \textcolor{red}{#1}}} % Enables comments in red on margin
\usepackage{mathtools}
\usepackage{amsmath}
\DeclarePairedDelimiter\abs{\lvert}{\rvert}%
\usepackage{cancel}
\usepackage{minted}
\usepackage{float}
%-------------------------------

%----------------------------------------------------------------------------------------

%-------------------------------
%	ENVIRONMENT SECTION
%-------------------------------
\pagestyle{fancy}
\usepackage{mdframed}


\newenvironment{problem}[2][Problem]
    { \begin{mdframed}[backgroundcolor=gray!20] \textbf{#1 #2} \\}
    {  \end{mdframed}}

% Define solution environment
\newenvironment{solution}
    {\textit{Solución}}
    {}


%-------------------------------------------------------------------------------------------
%	CUSTOM COMMANDS
%-------------------------------
\newcommand{\gaussian}{\frac{1}{\sigma\sqrt{2\pi}}\exp\left(- \frac{(x-\mu)^2}{2\sigma^2}\right)}
\newcommand{\R}{\mathbb R}

\newtheorem{thm}{Teorema}
\newtheorem{lema}{Lema}

\newtheorem{ndef}{Definición}

\def\inline{\lstinline[basicstyle=\ttfamily,keywordstyle={}]}


\begin{document}

%-------------------------------
%	TITLE SECTION
%-------------------------------

\fancyhead[C]{}
\hrule \medskip % Upper rule
\begin{minipage}{0.295\textwidth}
\raggedright
\footnotesize
Francisco Javier Sáez Maldonado \hfill\\
77448344F \hfill\\
franciscojavier.saez@estudiante.uam.es
\end{minipage}
\begin{minipage}{0.4\textwidth}
\centering
\large
Take Home Exam 2\\
\normalsize
Convex Optimization\\
\end{minipage}
\begin{minipage}{0.295\textwidth}
\raggedleft
\today\hfill\\
\end{minipage}
\medskip\hrule
\bigskip

%-------------------------------
%	CONTENTS
%-------------------------------

\begin{problem}{1}
We have worked out the elementary vision of Lagrange multipliers, assuming that , from \(g(x,y) = 0\), we can find a function \(y = h(x)\) such that \(g(x,h(x)) = 0\).\\

But sometimes, what we get is that there is an \(h\) such that \(g(x,h(x)) = 0\). Rewrite the Lagrange multiplier analysis in the lecture slides under this assumption.
\end{problem}

Consider \(f,g:\R^2 \to \R\), and the following minimization problem:
\begin{equation}
\min f(x,y) \ \ \text{s.t.  } g(x,y) = 0 
\end{equation}

Now, we can use the \textbf{implicit function theorem} to find a dependence between the variables of the restriction. This theorem (not completely formally) \textbf{states} the following: let \(g:\R^{n+m} \to \R^m\) be a continuously differentiable function, \((x,y) \in \R^{n+m}\) such that \(g(x,y) = 0\). If the jacobian with respect to the variables in \(y\) is invertible, then there exists an open subset \(U\) such that \(h(x) = y\) and \(g(x,h(x)) = 0\) for all \(x \in U\). \\

\textbf{Assuming} that the \textbf{conditions for this theorem are matched}, we can apply it to the \textbf{jacobian with respect to the variables in y} to obtain an \(U'\) where \(h(y) = x\) and \(f(h(y),y) = 0\) for all \(y \in U\). Thus, we can write:
\[
f(x,y) = f(h(y),y) =  \psi(y) 
\]
The, we can keep the procedure as it is done in the slides. Let us see this:\\

Consider that \(y^*\) is a minimum with \(x^* = h(y^*)\). Then, we have:
\[
0 = \psi'(y^*) = \frac{\partial f}{\partial x} (x^*, y^*)h'(y^*) + \frac{\partial f}{\partial y}(x^*, y^*).
\]
Using that \((x^*,y^*)\) is a minimum and that \(g(h(y),y) = 0\), we have that:
\[
0 = \frac{\partial g}{\partial x}(x^*, y^*)h'(y^*) + \frac{\partial g}{\partial y}(x^*, y^*) \implies h'(y^*) = \frac{a}{b}   
\]




\begin{problem}{2}
We want to solve the following constrained restriction problem:
\begin{align*}
  \min \quad       & x^{2} + 2xy + 2y^2 - 3x + y \\
  \text{s.t} \quad & x + y = 1            \\
                   & x,y \geq 0.
\end{align*}
Argue first that \(f\) is convex and then:
\begin{enumerate}
  \item Write its Lagrangian with \(\alpha,\beta\) the multipliers of the inequality constraints.
  \item Write the KKT conditions.
  \item Use them to solve the problem. For this consider separately the \((\alpha = \beta = 0)\), \((\alpha > 0, \beta = 0)\), \((\alpha = 0, \beta > 0)\), \((\alpha > 0, \beta > 0)\) cases.
\end{enumerate}
\end{problem}

\begin{problem}{3}
Let \(f: S \subset \R^d \to \R\) be a convex function on the convex set \(S\) and we exetnd it to an \(\tilde f : \R^d \to \R\) as:
\[
\tilde f(x) = \begin{cases}
f(x) & \text{ if } x \in S\\
+\infty & \text{ if } i \notin S 
\end{cases}
\]
Show that \(\tilde f\) is a convex function on \(\R^d\). Assume that \(a+\infty = \infty\) and \(a\cdot \infty = \infty\) for \(a > 0\).
\end{problem}

Consider \(x,x' \in \R^d\). We can consider two cases:

\begin{enumerate}
\item If \(x,x' \in S\), since \(S\) is convex, \(\lambda x + (1-\lambda)x' \in S\) for any \(\lambda \in [0,1]\). Also, we know that in \(S\) we have \(f(x) = \tilde f(x)\) and the same happens for \(x'\). Applying this and that \(f\) is convex in \(S\):
\[
\tilde f(\lambda x + (1-\lambda)x')    
\]
\end{enumerate}

\end{document}
